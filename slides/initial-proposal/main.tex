\documentclass[10pt]{beamer}

\RequirePackage{fontawesome}

\title{Group 3}
\author{Aamir Sangey, Manmilan Singh, Peter Vu}

\usetheme{CambridgeUS}
\usecolortheme{beaver}

\begin{document}

\begin{frame}

\frametitle{Initial Project Proposal}

\begin{itemize}
    \item Paper: \href{https://arxiv.org/abs/2111.09886}{\underline{\textit{SimMIM: A Simple Framework for Masked Image Modeling}} \faExternalLink}
    \begin{itemize}
        \item Accepted to CVPR 2022. Authored by Microsoft. \href{https://github.com/microsoft/SimMIM}{\underline{GitHub} \faExternalLink}
    \end{itemize}
    \item Description: A simplified framework for masked image modeling.
    \begin{itemize}
        \item \textbf{Masked image modeling}: Predicting masked out regions of an image.
        \item Prior frameworks for MIM required complex designs.
        \item SimMIM yields competitive/SOTA benchmarks while being simpler.
    \end{itemize}
    \item New dataset: \href{https://www.tensorflow.org/datasets/catalog/chexpert}{\underline{CheXpert} \faExternalLink}
    \begin{itemize}
        \item 224,316 chest radiographs of 65,240 patients.
        \item 14 observations per image, each positive, negative, or uncertain.
    \end{itemize}
    \item Core technical challenge:
    \begin{itemize}
        \item \textit{SimMIM}, as with prior MIM frameworks, is trained on ImageNet-1K.
        \item However, radiographs are significantly less diverse than ImageNet images.
        \item Furthermore, features of a positive observation may be strongly localized in specific image regions, but labels are ignored in pre-training. So, the model may primarily learn global anatomical structure rather than pathology-relevant representations.
        \item The core challenge is therefore determining whether SimMIM can perform self-supervised representation learning of radiographs and whether these representations can be efficiently fine-tuned to accurately classify clinical observations.
    \end{itemize}
\end{itemize}

\end{frame}

\end{document}
